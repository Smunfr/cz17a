%scrrprt ist eine Klasse für Berichte und längere Arbeiten
\documentclass[11pt,a4paper]{scrreprt}
%ermöglicht die Eingabe von Umlauten usw. ohne Codierung
\usepackage[utf8]{inputenc}
%Schriften werden mit einer passenden Kodierung für europäische Zeichen ausgegeben
\usepackage[T1]{fontenc}
%Passt Dokumentelemente an die Konventionen der deutschen Sprache (neue Rechtschreibung) an, z.b. Datumsangaben, Silbentrennung
\usepackage[ngerman]{babel}
\title{Lastenheft}
\author{Lisa Vogelsberg}
\date{29.11.2017}

\begin{document}
\tableofcontents

%Bei reports beginnt die Zählung von sections mit 0. Daher wurde hier chapter gewählt. Danach kann section gewählt werden, danach subsection und subsubsection
\chapter{Ausgangssituation}
%LV: Übernommen aus der Themenbeschreibung im OLAT
Wissensspiele erfreuen sich im privaten Sektor höchster Popularität (Wer wird Millionär, etc.). Im Bereich der Weiterbildung und des Wissensaustausches in Unternehmen werden diese bisher jedoch nur selten eingesetzt. Als Erweiterung zu klassischen E-Learning-Anwendungen haben Wissensspiele ein erhebliches Potential. Im Rahmen des aktuellen Projektes SB:Digital (http://sbdigital.infai.org/) werden neue Formen des gamifizierten Lernens und des Wissensaustausches entwickelt und erprobt.

\chapter{Zielsetzung und Produkteinsatz}
\section{Vision}
%LV: Ich krieg das nicht schön bzw. prägnant ausformuliert, deswegen erstmal so. Vielleicht kann das jemand besser ausdrücken.
Mitarbeiter von Unternehmen sollen mit Spaß und Unterhaltung dazu motiviert werden, sich an der betrieblichen Weiterbildung zu beteiligen. Wissen soll vermittelt werden, ohne dass es als Last empfunden wird. Im besten Fall wollen die Mitarbeiter sich an der Weiterbildung über das gegebene notwendige Maß hinaus beteiligen.
\section{Zielsetzung}
Die zu entwickelnde Software soll unter Verwendung verschiedener Techniken der Gamification die betriebliche Weiterbildung und den Wissensaustausch unterstützen.
Als Kernkomponente soll ein Quiz mit einer Frage, vier Auswahlmöglichkeiten und einer richtigen Antwort erstellt werden, an dem Gruppen von mindestens 5 Personen teilnehmen.
Gewinnen soll dabei aber nicht immer nur derjenige, der am meisten weiß, sondern derjenige, der Wissen, Strategie und Zufall für sich zu nutzen weiß.
\section{Produkteinsatz}
Zielgruppe des Produkts sind Arbeitnehmer in Unternehmen verschiedener Branchen, die sich während ihrer Arbeitszeit mit dem Quiz beschäftigen. Die Teilnahme erfolgt anonym dahingehend, dass nicht festgestellt werden kann, welche reale Person gerade an dem Quiz teilnimmt.

\chapter{Funktionale Anforderungen}
%LV: Muss- und Kann-Ziele finde ich für die Anforderungen noch nicht ganz passend, da man zu schnell mit den Zielen verwechseln die kann, die man vorher festlegt und durch die sich dann die Anforderungen ergeben. Einteilung in Hauptanforderungen und optionale Anforderungen trifft es vielleicht schon eher, ist trotzdem noch nicht gut von mir gewählt (Hauptanforderungen meint nämlich eigentlich etwas anderes). Würde ich Dienstag gern besprechen, wenn wir bis dahin nicht noch eine bessere Bezeichnung gefunden haben.
\section{Hauptanforderungen}
%Erzeugt eine Liste ohne Nummerierung, bei dem das Wort in eckigen Klammern als Beschreibung dient und fett gedruckt wird. Neue Zeilen innerhalb eines items werden eingerückt.
\begin{description}
\item[/LF0010/] Die Auswertung von Spielergebnissen soll über ein Dashboard erfolgen
\item[/LF0020/] Client muss bei jedem Login vom Server wiedererkannt werden.
\item[/LF0030/] Fragen und spielerbezogene Daten müssen gespeichert werden können
\item[/LF0040/] ACL zum Administrieren
\end{description}
Gamification Elemente:
\begin{description}
\item[/LF0050/] Badges
\item[/LF0060/] Ranking-System
\end{description}
\section{Optionale Anforderungen}
\begin{description}
\item[/LFXXXX/]
\end{description}

\chapter{Nicht-funktionale Anforderungen}
\section{Hauptanforderungen}
\begin{description}
\item[/LL0010/]
\item[/LL0020/]
\item[/LL00XX/]
\end{description}
\section{Optionale Anforderungen}
\begin{description}
\item[/LL0010/]
\item[/LL0020/]
\item[/LL00XX/]
\end{description}

\chapter{Qualitätsmatrix nach ISO 25010}
\begin{itemize}
\item
\end{itemize}

\chapter{Lieferumfang und Abnahmekriterium}
\begin{itemize}
\item
\end{itemize}

\chapter{Vorprojekt}
\begin{itemize}
\item
\end{itemize}

\chapter{Glossar}
\begin{itemize}
\item
\end{itemize}
\end{document}