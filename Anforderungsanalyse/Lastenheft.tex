%scrrprt ist eine Klasse für Berichte und längere Arbeiten
\documentclass[11pt,a4paper]{scrreprt}
%ermöglicht die Eingabe von Umlauten usw. ohne Codierung
\usepackage[utf8]{inputenc}
%Schriften werden mit einer passenden Kodierung für europäische Zeichen ausgegeben
\usepackage[T1]{fontenc}
%Passt Dokumentelemente an die Konventionen der deutschen Sprache (neue Rechtschreibung) an, z.b. Datumsangaben, Silbentrennung
\usepackage[ngerman]{babel}
\title{Lastenheft}
\author{Lisa Vogelsberg}
\date{29.11.2017}

\begin{document}
\tableofcontents
%Bei reports beginnt die Zählung von sections mit 0. Daher wurde hier chapter gewählt. Danach kann section gewählt werden, danach subsection und subsubsection
\chapter{Ausgangssituation}
Hier steht Text Text Text Text Text Text Text Text Text Text Text Text Text Text Text Text Text Text Text Text Text Text Text Text Text Text Text Text Text Text Text Text Text Text Text Text Text Text Text Text Text Text Text Text Text Text Text Text Text Text Text Text Text Text Text Text Text Text Text Text Text Text Text Text Text Text Text Text Text Text Text Text Text Text Text Text Text Text Text Text Text Text Text Text Text Text Text Text Text Text Text Text Text Text Text Text Text Text Text Text Text Text Text Text Text Text Text Text Text Text Text Text Text Text
\chapter{Zielsetzung und Produkteinsatz}
\chapter{Funktionale Anforderungen}
\chapter{Nicht-funktionale Anforderungen}
\chapter{Qualitätsmatrix nach ISO 25010}
\chapter{Lieferumfang und Abnahmekriterium}
\chapter{Vorprojekt}
\chapter{Glossar}
\end{document}