\documentclass[11pt,a4paper]{scrreprt}
%Umlaute ohne Codierung
\usepackage[utf8]{inputenc}
%europäische Zeichen
\usepackage[T1]{fontenc}
%neue Rechtschreibung
\usepackage[ngerman]{babel}
\usepackage{scrpage2}
\pagestyle{scrheadings}
\clearscrheadfoot

%kein Seitenvorschub
\makeatletter
\newcommand\Chapter{%
                    \par\vspace{0.01cm}% anpassen
                    \global\@topnum\z@
                    \@afterindentfalse
                    \secdef\@chapter\@schapter}
\makeatother

\begin{document}
\begin{titlepage}
	\centering
	{\scshape\LARGE Universität Leipzig \par}
	\vspace{1cm}
	{\scshape\Large Softwaretechnik-Praktikum \par}
	\vspace{2cm}
	{\huge\bfseries Projektplan Gruppe cz17a - Gamification \par}
	\vspace{2cm}
	{\Large\itshape Lisa Vogelsberg, Felix Fink, Michael Fritz, Thomas Gerbert, Steven Lehmann, Fabian Ziegner, Willy Steinbach, 	Christian Schlecht \par}
	\vfill
	supervised by \par
	Dr.~Christian \textsc{Zinke}, Julia \textsc{Friedrich}, Christian \textsc{Frommert}
	\vfill
	{\large \today \par}
\end{titlepage}
\tableofcontents

\Chapter{Arbeitspakete}
\section{Arbeitspaket 1 - Vorprojekt}
Die Umsetzung des Vorprojekts leitet das Projekt als Grundgerüst ein. \\ \\
\textit{Wichtigkeit: 25\%, Zeit: 20\%} \\
Das Vorprojekt besteht im Wesentlichen aus einem Grundgerüst der Anwendung. Es werden die Datenbank, die Basis für eine Server und einen Client, eine REST-Schnittstelle auf dem Server und das AdminPanel begonnen. Über das AdminPanel sollen bereits Fragen inkl. flexibler Anzahl von Antworten in die Datenbank eingelesen werden können. Ein prototypischer Quizmechanismus stellt die Fragen per Datenbankabfrage zur Verfügung, sodass der Spieler bereits eine davon auswählen kann und eine Rückmeldung darüber erhält, ob seine gegebene Antwort richtig war.\\
Zugeordnete Punkte des Lastenhefts:
\begin{itemize}
\item LF0510
\end{itemize}

\section{Arbeitspaket 2 - Spieler und Quiz}
Im Arbeitspaket 2 werden Spieleraccounts eingeführt und das Quiz erweitert. \\ \\
\textit{Wichtigkeit: 30\%, Zeit: 25\%} \\
Im zweiten Arbeitspaket werden die Spieleraccounts eingeführt und das Quiz erweitert. Demzufolge wird die Registierung, An - und Abmeldung von Spielern, so wie die Anzeige der Spieleraccounts implementiert. Das Quiz soll um die wesentlichen Funktionalitäten erweitert werden: mehrere Fragen pro Quizrunde, Punkteausschüttung, Ergebnisansicht am Ende der Runde.\\ 
Zugeordnete Punkte des Lastenhefts:
\begin{itemize}
\item LF0010 - LF0050
\item LF0110 - LF0130
\item LF0210 - LF0240
\item LF1210 - LF1250
\end{itemize}


\section{Arbeitspaket 3 - Quizfinalisierung}
Das dritte Arbeitspaket finalisiert das eigentliche Quiz. \\ \\
\textit{Wichtigkeit: 15\%, Zeit: 15\%} \\
Das dritte Arbeitspaket stellt die Finalisierung des Quiz dar. Dazu erfolgt die Implementierung des Jackpots  und der Fragencharakteristika (Fragenauswahl, Auftrittswahrscheinlichkeit, dynamischer Schwierigkeitsgrad, Punktzahl, \dots  .\\
Zugeordnete Punkte des Lastenhefts:
\begin{itemize}
\item LF2210 - LF2250
\item LF3210 - LF 3260
\item LF0520
\end{itemize}

\section{Arbeitspaket 4 - Administration}
Arbeitspaket 4 beschäftigt sich mit der Umsetzung der Administrationsoberfläche und Implementierung der ACL. \\ \\
\textit{Wichtigkeit:15\%, Zeit: 15\%} \\
Das Arbeitspaket 4 bietet die wichtigen Einstellungsmöglichkeiten für den Administrator der App, indem die Administrationsoberfläche aus dem Vorprojekt aufgegriffen wird. Es werden z.B. Fragenanzahl, Fragenverwaltung und - import einstellbar sein. Durch die ACL wird nur dem Administrator selbst Zugang zu dieser Oberfläche gewährt.\\
Zugeordnete Punkte des Lastenhefts:
\begin{itemize}
\item LF0310
\end{itemize}

\section{Arbeitspaket 5 - Dashboard}
Im Fokus steht die Umsetzung des kompletten Dashboards und damit Finalisierung der Muss-Ziele. \\ \\
\textit{Wichtigkeit: 25\%, Zeit: 15\%} \\
Im letzten umzusetzendem Arbeitspaket wird das gesamte Dashboard erstellt. Dies umfasst das Aufrufen des persönlichen Dashboards für den Spieler, sowie die Errechnung und Anzeige aller relevanten Dashboard-Daten.\\
Zugeordnete Punkte des Lastenhefts:
\begin{itemize}
\item LF0410 - LF0420
\end{itemize}

\end{document}