\documentclass[11pt,a4paper]{scrreprt}
%Umlaute ohne Codierung
\usepackage[utf8]{inputenc}
%europäische Zeichen
\usepackage[T1]{fontenc}
%neue Rechtschreibung
\usepackage[ngerman]{babel}
\usepackage{scrpage2}
\pagestyle{scrheadings}
\clearscrheadfoot

%kein Seitenvorschub
\makeatletter
\newcommand\Chapter{%
                    \par\vspace{0.01cm}% anpassen
                    \global\@topnum\z@
                    \@afterindentfalse
                    \secdef\@chapter\@schapter}
\makeatother

\begin{document}
\begin{titlepage}
	\centering
	{\scshape\LARGE Universität Leipzig \par}
	\vspace{1cm}
	{\scshape\Large Softwaretechnik-Praktikum \par}
	\vspace{2cm}
	{\huge\bfseries Projektplan Gruppe cz17a - Gamification \par}
	\vspace{2cm}
	{\Large\itshape Lisa Vogelsberg, Felix Fink, Michael Fritz, Thomas Gerbert, Steven Lehmann, Fabian Ziegner, Willy Steinbach, 			Christian Schlecht \par}
	\vfill
	supervised by \par
	Dr.~Christian \textsc{Zinke}, Julia \textsc{Friedrich}, Christian \textsc{Frommert}
	\vfill
	{\large \today \par}
\end{titlepage}
\tableofcontents

\Chapter{Gliederung}
Das Projekt soll sich in insgesamt vier Arbeitpakete einteilen, wobei ein zusätzliches fünftes Arbeitspaket mit Kann-Zielen optional ist.
Die Arbeitspakete bauen logisch aufeinander auf, weshalb sich gewissermaßen 'automatisch' eine Reihenfolge festlegt. Folgende Arbeitspakete sollen umgesetzt werden:
\begin{description}
\item[Arbeitspaket 1 - Vorprojekt] \ \\
Die Umsetzung des Vorprojekts leitet das Projekt als Grundgerüst ein.
\item[Arbeitspaket 2 -  Spieler und Quiz] \ \\
Im Arbeitspaket 2 werden Spieleraccounts eingeführt und das Quiz erweitert.
\item[Arbeitspaket 3 -  Quiz und Administration] \ \\
Das dritte Arbeitspaket finalisiert das eigentliche Quiz und erweitert die Administrationsmöglichkeiten.
\item[Arbeitspaket 4 - Dashboard] \ \\
Im Fokus steht die Umsetzung des kompletten Dashboards und damit Finalisierung der Muss-Ziele
\item[optionales Arbeitspaket 5 - Kann-Ziele] \ \\
Im optionalen Arbeitspaket steht die Umsetzung einiger Kann-Ziele
\end{description}



\Chapter{Arbeitspakete}



\end{document}