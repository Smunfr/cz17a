\documentclass[11pt,a4paper]{scrreprt}
%Umlaute ohne Codierung
\usepackage[utf8]{inputenc}
%europäische Zeichen
\usepackage[T1]{fontenc}
%neue Rechtschreibung
\usepackage[ngerman]{babel}
\usepackage{scrpage2}
\pagestyle{scrheadings}
\clearscrheadfoot

%kein Seitenvorschub
\makeatletter
\newcommand\Chapter{%
                    \par\vspace{0.01cm}% anpassen
                    \global\@topnum\z@
                    \@afterindentfalse
                    \secdef\@chapter\@schapter}
\makeatother

\begin{document}
\begin{titlepage}
	\centering
	{\scshape\LARGE Universität Leipzig \par}
	\vspace{1cm}
	{\scshape\Large Softwaretechnik-Praktikum \par}
	\vspace{2cm}
	{\huge\bfseries Releaseplan Gruppe cz17a - Gamification \par}
	\vspace{2cm}
	{\Large\itshape Lisa Vogelsberg, Felix Fink, Michael Fritz, Thomas Gerbert, Steven Lehmann, Fabian Ziegner, Willy Steinbach, 	Christian Schlecht \par}
	\vfill
	supervised by \par
	Dr.~Christian \textsc{Zinke}, Julia \textsc{Friedrich}, Christian \textsc{Frommert}
	\vfill
	{\large \today \par}
\end{titlepage}
\tableofcontents

\Chapter{Arbeitpaket 1 - Vorprojekt}
\textit{Release: 22.01.18} \\
Das Vorprojekt bildet das Grundgerüst der Anwendung. Es wird eine Server-Client Architektur entwickelt. Zusätzlich werden eine PostgreSQL Datenbank und eine REST-Schnittstelle erstellt und über den Praktikumsserver angebunden.

\section{Datenbank}
\begin{itemize}
\item Die Datenbank wird modelliert, erstellt und auf den Praktikumsserver geladen.
\item Aus der Datenbank werden die Fragen ausgelesen.
\end{itemize}

\section{Server}
\begin{itemize}
\item der Server übernimmt die zentrale Verarbeitung des Quiz
\item der Server händelt alle Verbindungen zu den Clients, im Rahmen des Vorprojekts beschränkt sich dies auf einen Client
\item über eine REST-Schnittstelle werden alle Anfragen an die Datenbank gestellt
\end{itemize}

\section{Client}
\begin{itemize}
\item der Client erhält vom Server eine Frage inkl. der Antworten
\item das User-Interface ermöglicht die Darstellung der Fragen und Antworten
\item der User kann eine Frage auswählen und erhält eine visuelle Rückmeldung, ob die Frage richtig beantwortet wurde.
\end{itemize}

\section{Admin-Panel}
\begin{itemize}
\item das Admin-Panel wird mittels einer Weboberfläche realisiert.
\item es können Fragen inkl. Antworten im CSV-Format in die Datenbank eingelesen werden.
\end{itemize}


\Chapter{Arbeitspaket 2 - Spieler und Quiz}
\textit{Release: 19.02.2018} \\
Im zweiten Release werden Spieleraccounts eingeführt und das Quiz erweitert.

\section{Spieleraccounts}
\begin{itemize}
\item Ein Spieler kann sich unter Angabe von Nutzername, E-Mail und Passwort registrieren.
\item Er kann sich einloggen, wieder abmelden, das Passwort bei Vergessen anfordern ändern und seine Accountübersicht einsehen. 
\item Der Account kann dauerhaft (d.h. endgültig) gelöscht werden.
\end{itemize}

\Chapter{Arbeitspaket 3 - Quizfinalisierung}
\textit{Release: 19.03.18} \\
Das dritte Arbeitspaket finalisiert das eigentliche Quiz.
\begin{itemize}
\item die zusammenfassende Ergebnisübersicht wird um ein anonymes Leaderboard der Runde erweitert.
\end{itemize}

\section{Quizerweiterung}
\begin{itemize}
\item Das Quiz wird auf mehrere Spieler (mind. 5 pro Runde) erweitert.
\end{itemize}
\subsection{Quizablauf}
\begin{itemize}
\item Eine Quizrunde wird auf mehrere Fragen aufeinanderfolgend erweitert
\item Nach Beendigung der Runde erfolgt eine zusammenfassende Ergebnisübersicht
\end{itemize}
\subsection{weitere Quizeigenschaften}
\begin{itemize}
\item Das Quiz wird um die Punkteausschüttung erweitert. Punkte werden für richtige Antworten vergeben.
\end{itemize}

\section{Jackpot}
\begin{itemize}
\item der Jackpot füllt sich durch die Punkte der Spieler, falls diese eine Frage falsch beantworten.
\item eine Jackpot-Frage kann zufällig ausgelöst werden. Dann wird um die gesamte Punktzahl im Jackpot gespielt. Bei richtiger Antwort ist ein Anteil des Jackpots garantiert.
\item die Wahrscheinlichkeit des Auftretens einer Jackpot-Frage steigt mit zunehmend falsch beantworteten Fragen.
\item Initial und nach Ausschüttung des Jackpots wird dieser mit einer gewissen Anzahl an Punkten gefüllt.
\end{itemize}

\section{weitere Frageneigenschaften}
\begin{itemize}
\item Fragen werden generell zufällig aus dem Fragenpool ausgewählt. Je häufiger eine Frage jedoch falsch beantwortet wird, desto größer wird die Wahrscheinlichkeit des Auftretens.
\item der Fragen wird bei Einfügung ins System ein statischer Schwierigkeitsgrad mit übergeben. Anhand der Beantwortungsstatistiken (Richtig/Falsche beantwortet) wird dieser dynamisch angepasst.
\item In Abhängigkeit des Schwierigkeitsgrades erhält der Punktwert einer Frage einen Multiplikator.
\item Ein Import von Excel-Dateien für Fragen wird an das Admin-Panel angebunden.
\end{itemize}

\Chapter{Arbeitspaket 4 - Administration}
\textit{Release: 02.04.2018} \\
Arbeitspaket 4 beschäftigt sich mit der Vervollständigung der Administrationsoberfläche und Einbindung der ACL.
\begin{itemize}
\item Es werden zwei Kontrollzustände in der ACL eingerichtet: Ein Administrator mit uneingeschränkten Rechten und ein User, welche das Spiel spielen und seine persönlichen Daten einsehen kann.
\item Dem Administrator werden alle Einstellungsmöglichkeiten über einen externen Webservice zur Verfügung gestellt (Admin-Panel aus dem Vorprojekt).
\item Er kann Fragen importieren/bearbeiten/löschen, die Rundenanzahl einstellen und Userrechte bearbeiten. Zusätzlich kann er strategische Modifikatoren, Badges, freischaltbare Inhalte und Items erstellen/bearbeiten/löschen, falls diese umgesetzt werden.
\end{itemize}

\Chapter{Arbeitspaket 5 - Dashboard}
\textit{Release: 16.04.2018} \\
Das letzte Arbeitspaket setzt als letzten großen Aspekt das Dashboard um.
\section{Dashboard}
\begin{itemize}
\item alle Dashboard-Daten müssen anhand der gespeicherten Daten der Datenbank errechnet werden.
\item das Dashboard soll für den Benutzer jederzeit aufrufbar sein.
\item die Darstellung muss möglichst übersichtlich sein, sodass die Daten schnell überblickt werden können.
\item angezeigt werden sollen z.B.:
\begin{itemize}
\item gesamte/durchschnittliche Spielzeit
\item Anzahl gespielter Runden
\item Prozentsatz richtiger Antworten
\item durchschnittliche, höchste, gesamte Punktzahl
\item \dots
\end{itemize}
\end{itemize}

\section{Abschlusstests}
Vor Abgabe des finalen Release werden noch einmal umfangreiche Abschlusstests durchgeführt. Dabei werden sowohl Unit-Tests als auch manuelle Tests so breit abdeckend wie möglich ausgeführt. Optional könnten Beta-Tester (unbeteiligte Dritte) einige Runden spielen, um eventuelle Bugs aufzudecken, welche in der letzten Iteration zu beheben sind.

\end{document}